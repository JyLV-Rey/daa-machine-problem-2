\subsection{ Bubble Sort }

% DESCRIPTION
\subsubsection{Description}

The optimization above features \textbf{bidirectional pass} and \textbf{dynamic boundary shrinking}. In this design, the algorithm alternates between a forward pass—moving the largest unsorted element to the end—and a backward pass—pushing the smallest element to the start. This dual-direction traversal allows the sorting to proceed simultaneously from both ends of the array, effectively halving the number of required iterations in many practical cases.

Moreover, the optimization uses \texttt{lastSwapped} variable to record the index at which the most recent swap occurred during each pass. After each forward or backward pass, the boundaries of the unsorted portion (\texttt{start} and \texttt{end}) are updated accordingly. The dynamic boundary shrinking minimizes redundant comparisons by ensuring that already sorted portions of the array are not reprocessed in subsequent iterations.

%% PSEUDOCODE
\subsubsection{Pseudocode}
Given an array $(Arr)$ of numbers with size $n$, below is the optimized bubble sort algorithm:

\begin{algorithm} [H]
    \caption{Optimized Bubble Sort (Ascending Order)}
    \begin{algorithmic}
    \Procedure{OptimizedBubbleSort}{$Arr$}
        \State{} $start \gets 0$ \Comment{Start index for unsorted portion}
        \State{} $end \gets \text{length}(Arr) - 1$ \Comment{End index for unsorted portion}
        \While{$start < end$}
            \State{} $lastSwapped \gets 0$
            \Comment{Forward pass: move largest element to the right}
            \For{$i \gets start$ \textbf{to} $end - 1$}
                \If{$Arr[i] > Arr[i + 1]$}
                    \State{} swap $Arr[i]$ and $Arr[i + 1]$
                    \State{} $lastSwapped \gets i$
                \EndIf{}
            \EndFor{}
            \State{} $end \gets lastSwapped$
            \If{$start \geq end$}
                \State{} \textbf{break} \Comment{Early exit: array is sorted}
            \EndIf{}
            \State{} $lastSwapped \gets 0$
            \Comment{Backward pass: move smallest element to the left}
                \For{$i \gets end - 1$ \textbf{downto} $start$}
                    \If{$Arr[i] > Arr[i + 1]$}
                        \State{} swap $Arr[i]$ and $Arr[i + 1]$
                        \State{} $lastSwapped \gets i + 1$
                    \EndIf{}
                \EndFor{}
            \State{} $start \gets lastSwapped$
        \EndWhile{}
    \EndProcedure{}
\end{algorithmic}
\end{algorithm}

%% PSEUDOCODE DESCRIPTION

Another enhancement is the inclusion of an \textbf{early termination condition} that checks whether \texttt{start} is greater than or equal to \texttt{end} (\texttt{start}~$\geq$~\texttt{end}) after a pass. If this condition is met, it implies that all elements are in sorted order, allowing the algorithm to terminate immediately. 

In contrast, the traditional bubble sort algorithm lacks both of these optimizations. It neither tracks swap positions nor adjusts the sorting bounds dynamically. As a result, it continues to iterate through the full unsorted portion even if no swaps occur, leading to poor performance on sorted or nearly sorted arrays. All passes are fixed in length and direction, contributing to its inefficiency in practical use.

Although both versions share the same asymptotic worst-case time complexity of \(O(n^2)\), the optimized bubble sort significantly outperforms the traditional approach in real-world scenarios. It reduces the total number of comparisons, limits swap operations, and speeds up convergence on partially sorted datasets.

