% DESCRIPTION
\subsection{Knapsack}
\subsubsection{Description}

The 0/1 Knapsack Problem using Branch and Bound is an optimization approach to solving the classic knapsack problem, where the goal is to maximize the total value of items placed in a knapsack without exceeding its weight capacity. Each item can either be included or excluded—hence the ``0/1'' nature of the problem. The Branch and Bound method systematically explores subsets of items using a decision tree, where each node represents a choice to include or exclude an item. It avoids unnecessary computation by pruning branches that cannot yield better results than the current best solution.

To do this, the algorithm calculates an upper bound at each node, estimating the maximum possible value that can still be achieved from that point. This bound is usually determined using a greedy approach that allows fractional items to fill the remaining capacity (only for estimation). If a node`s bound is less than or equal to the best solution found so far, it is discarded. This bounding technique greatly reduces the number of paths explored compared to brute force. Although the worst-case time complexity is still exponential, Branch and Bound performs efficiently in practice for moderately sized problems.

\subsubsection{Pseudocode}
\begin{algorithm} [H]
    \caption{Knapsack using Branch and Bound}
    \begin{algorithmic}
    \Function{KnapsackBranchBound}{$weights, values, capacity$}
        \State{} Sort items by decreasing value-to-weight ratio
        \State{} Initialize a max priority queue $Q$
        \State{} Create root node with level = $-1$, profit = $0$, weight = $0$
        \State{} Set root.bound = \Call{CalculateBound}{root}
        \State{} $Q$.enqueue (root)
        \State{} $maxProfit \gets 0$
        \While{$Q$ is not empty}
            \State{} $node \gets Q$.dequeue () \Comment{Node with max bound}
            \If{$node.bound \leq maxProfit$}
                \State{} \textbf{continue}
            \EndIf{}
            \State{} Create $child1$ by including next item
            \If{$child1.weight \leq capacity$ \textbf{and} $child1.profit > maxProfit$}
                \State{} $maxProfit \gets child1.profit$
            \EndIf{}
            \State{} Compute $child1.bound$
            \If{$child1.bound > maxProfit$}
                \State{} $Q$.enqueue ($child1$)
            \EndIf{}
            \State{} Create $child2$ by excluding next item
            \State{} Compute $child2.bound$
            \If{$child2.bound > maxProfit$}
                \State{} $Q$.enqueue ($child2$)
            \EndIf{}
        \EndWhile{}
        \State{} \Return{}{} $maxProfit$
    \EndFunction{}
    
    \Function{CalculateBound}{$node$}
        \If{$node.weight \geq capacity$}
            \State{} \Return{} $0$
        \EndIf{}
        \State{} $profitBound \gets node.profit$
        \State{} $j \gets node.level + 1$
        \State{} $totalWeight \gets node.weight$
        \While{$j < n$ \textbf{and} $totalWeight + weights[j] \leq capacity$}
            \State{} $totalWeight \gets totalWeight + weights[j]$
            \State{} $profitBound \gets profitBound + values[j]$
            \State{} $j \gets j + 1$
        \EndWhile{}
        \If{$j < n$}
            \State{} $profitBound \gets profitBound + (capacity - totalWeight) \times \frac{values[j]}{weights[j]}$
        \EndIf{}
        \State{} \Return{} $profitBound$
    \EndFunction{}
    \end{algorithmic}
\end{algorithm}